%%%%%%%%%%%%%%%%%%%%%%%% FRAME 0 %%%%%%%%%%%%%%%%%%%%%%%%%%%%%%%
\begin{frame}{Noise Distributions at \SI{\sim10}{\mega\hertz\per\centi\meter^2}}
 
	\begin{figure}\vspace*{-15pt}
		\centering
		\subfigp{PD0}{.5}{scCVD with \SI{6}{\deci\bel} attenuation}
		\subfigp{PD1}{.5}{pCVD}
	\end{figure}\vspace*{-10pt}

	\begin{itemize} \itemfill
		\item noise distribution agrees well with Gaussian even at high rates
		\item extract noise by taking the sigma of the Gaussian fit
		\item noise similar for scCVD and pCVD diamond
	\end{itemize}
 
\end{frame}
%%%%%%%%%%%%%%%%%%%%%%%% FRAME 1 %%%%%%%%%%%%%%%%%%%%%%%%%%%%%%%
\begin{frame}{Signal Distributions at \SI{\sim10}{\mega\hertz\per\centi\meter^2}}
 
	\begin{figure}\vspace*{-15pt}
		\centering
		\subfigp{SD0}{.5}{scCVD with \SI{6}{\deci\bel} attenuation}
		\subfigp{SD1}{.5}{pCVD}
	\end{figure}\vspace*{-10pt}

	\begin{itemize} \itemfill
		\item signal gets corrected by the mean of the noise (baseline offset)
		\item pCVD signal smaller and smeared by different regions in the diamond
	\end{itemize}
 
\end{frame}
%%%%%%%%%%%%%%%%%%%%%%%% FRAME 2 %%%%%%%%%%%%%%%%%%%%%%%%%%%%%%%
\begin{frame}{Signal Maps}
 
	\begin{figure}\vspace*{-15pt}
		\centering
		\subfigp{SM0}{.5}{scCVD with \SI{6}{\deci\bel} attenuation}
		\subfigp{SM1}{.5}{pCVD}
	\end{figure}
	
	\begin{itemize} \itemfill
		\item flat signal distribution in scCVD
		\item signal response depending on region in the pCVD
	\end{itemize}
 
\end{frame}
%%%%%%%%%%%%%%%%%%%%%%%% FRAME 3 %%%%%%%%%%%%%%%%%%%%%%%%%%%%%%%
\begin{frame}{Currents}
 
	\vspace*{-15pt}\fig{Currents.png}{.6}
	
	\begin{itemize} \itemfill
		\item typical rate scans for \SI{\sim30}{\hour} with rates up to \SI{\sim20}{\mega\hertz\per\centi\meter^2}
		\item beam induced current clearly visible
		\item low leakage currents (\SI{<30}{\nano\ampere}) at a bias voltage of \SI{-1000}{\volt} (\SI{2}{\volt\per\micro m})
	\end{itemize}
 
\end{frame}
%%%%%%%%%%%%%%%%%%%%%%%% FRAME 4 %%%%%%%%%%%%%%%%%%%%%%%%%%%%%%%
\begin{frame}{Rate Studies}

	
	\vspace*{-15pt}
	\figp{B2Oct151}{.65} 
	
	\begin{itemize}
		\item systematically checking several up and down scans
		\item pumping required in the beggging to reach stable pulse height
		\item random scans to rule out systematic effects 
	\end{itemize}
	
\end{frame}
%%%%%%%%%%%%%%%%%%%%%%%% FRAME 5 %%%%%%%%%%%%%%%%%%%%%%%%%%%%%%%
\begin{frame}{Rate Studies in Non-Irradiated scCVD}

	
	\vspace*{-15pt}
	\figp{S129Scans1}{.65} 
	
	\begin{itemize}
		\item scCVD as reference in all beam tests
		\item all scans scaled to 1
		\item scCVD diamond shows now rate dependence within the measurement precision
	\end{itemize}
	
\end{frame}
%%%%%%%%%%%%%%%%%%%%%%%% FRAME 6 %%%%%%%%%%%%%%%%%%%%%%%%%%%%%%%
\begin{frame}{Rate Studies in Irradiated pCVD}

	\vspace*{-15pt}
	\figp{B2Scans1}{.65}
	
	\begin{itemize}
		\itemfill
		\item all scans scaled to 1
		\item pulse height very stable after irradiation
		\item noise stays the same of: $\upsigma \approx$ \SI{4.9}{au}
% 		\item signal degradation due to radiation damage (no absolute calibration)
	\end{itemize}

\end{frame}